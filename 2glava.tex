\section{Таможенная политика РФ}

Таможенная политика является неотъемлемой частью внутренней и внешней политики государства, представляет систему осуществляемых в национальных интересах мер воздействия на деятельность участников ВЭД в целях обеспечения наиболее эффективного использования инструментов таможенного контроля и регулирования товарообмена на таможенной территории страны, участия в реализации торгово-политических задач по защите внутреннего рынка, стимулирования развития национальной экономики, содействия проведению структурной перестройки и других задач экономической политики.

Таможенная политика --- это разработка государством тарифных и нетарифных методов, инструментов регулирования внешней торговли, таможенных правил и процедур, регулирующих движение товаров, капиталов, услуг, интеллектуальных и трудовых ресурсов через таможенную границу в интересах защиты и развития национальной экономики, пополнения доходов государственного бюджета, интеграции страны в мировую экономику и расширения внешнеэкономических связей.

Таможенная политика основывается на принципах единства, стабильность, независимости, открытости, равноправия, признания приоритета международных договоров и соглашений.

Субъектами таможенной политики являются законодательные и исполнительные органы государства, союзы предпринимателей, коммерческие банки, неправительственные и негосударственные международные организации.

Объектами таможенной политики выступают участники ВЭД, другие социальные группы, интересы которых затрагивает таможенная политика.

На содержание таможенной политики оказывают влияние следующие факторы:
\begin{enumerate}
	\item [1)] Политические:\\
	--- государственная политика;\\
	--- внутренняя политика;\\
	--- внешняя политика;\\
	--- экономическая политика;\\
	--- внешнеэкономическая политика;\\
	--- социальная политика.
	\item [2)] Экономические:\\
	--- темпы роста ВВП страны;\\
	--- динамика инвестиционной и инновационной активности, научно-техниче-ского потенциала;\\
	--- состояние и динамика материального производства и социально-культур-ной (непроизводственной) сферы;\\
	--- состояние финансовой системы;\\
	--- состояние кредитно-банковской системы;\\
	--- размеры и динамика теневого сектора экономики;\\
	--- размеры внешнего и внутреннего государственного долга;\\
	--- размеры и структура экспорта и импорта и др.
	\item [3)] Социальные:\\
	--- социальное положение населения страны;\\
	--- приоритеты социального развития;\\
	--- уровень безработицы и др.
	\item [4)] Внешние:
	--- состояние и тенденции развития международной торговли и туризма;\\
	--- участие страны в международных и 
\end{enumerate}
















