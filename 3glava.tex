\section{Задача}
Определите цену товара реализованного комиссионером,
участвующим в расчетах за наличный расчет.

Известно:\\
1. Цена передачи товара комитентом – 250 руб. (НДС).\\
2. Комиссионное вознаграждение – 70 руб.\\
3. НДС – 18\%

\subsection*{Решение}
1) Находим НДС в цене комитента:
\[ \dfrac{250}{118} \times 18 = 38\  \text{руб.}\] 

2) Находим цену комитента без НДС:
\[ 250 - 38 = 212 \  \text{руб.} \]

3) Находим цену комиссионера без НДС:
\[ 212 + 70 = 282\  \text{руб.} \]

4) Находим НДС в цене комиссионера:
\[ 282 \times 0,18 = 51 \  \text{руб.}  \]

5) Находим цену комиссионера с НДС:
\[ 282 + 51 = 333\  \text{руб.} \]

Ответ: цена товара, реализованного комиссионером 333 руб.