\section{Таможенная стоимость, методы ее определения.}

Вопрос определения и контроля таможенной стоимости ввозимых товаров в настоящее время является одним из самых актуальных, так как таможенная стоимость служит основой для начисления таможенных платежей.

Правильное определение таможенной стоимости является особо важным для таможенных органов, что связано с тем, что таможенные платежи составляют существенную часть федерального бюджета. Для предприятий вопрос определения таможенной стоимости играет ключевую роль, так как эффективность внешнеэкономической деятельности во многом зависит от уплачиваемых таможенных платежей \cite[с. 249--250]{mahovikova}.

Таможенная стоимость товаров, ввозимых на территорию таможенного союза (далее ТС), определяется в соответствии с международным договором государств --- членов ТС, регулирующим вопросы определения таможенной стоимости товаров, перемещаемых через границу. В РФ определяется в соответствии Соглашением между Правительством РФ, Правительством Республики Беларусь и Правительством Республики Казахстан от 25.01.2008 <<Об определении таможенной стоимости товаров, перемещаемых через таможенную границу Таможенного союза>> (ред. от 23.04.1012). Таможенная стоимость товаров, вывозимых с таможенной территории ТС, определяется в соответствии с законодательством государства --- члена ТС, таможенному органу которого производится декларирование товаров \cite[с. 486]{tolkushkin}.

Таможенная стоимость товара --- стоимость сделки с товаром, т.е. цена, фактически уплаченная или подлежащая уплате за эти товары при их продаже для вывоза на единую  таможенную территорию ТС. Таможенная стоимость --- это особая разновидность стоимости товара, в определении которой помимо продавца и покупателя участвует третья сторона --- таможенный орган.

Основу современной процедуры таможенной оценки товаров, принятой в большинстве стран мира, составляют ст. VII ГАТТ (General Agreement on Tariffs and Trade --- Генеральное соглашение по тарифам и торговле) и Соглашение по применению статьи VII ГАТТ. Наиболее точное определение таможенной стоимости дается в Соглашении по применению ст. VII ГАТТ \cite[с. 124]{novikova}.
\begin{quote}
	Таможенная стоимость товара --- цена, фактически уплаченная или подлежащая уплате за товары при продаже с целью экспорта в страну импорта, скорректированная с учетом установленных дополнительных начислений к этой цене.
\end{quote}

Необходимость унификации процессов определения таможенной стоимости была связана с тем, что длительное время в разных странах применяемые методы оценки таможенной стоимости значительно отличались. Не зная конечную цену продаваемого товара невозможно было определить экономическую эффективность сделки.

В 1950 г. в Брюсселе была заключена Конвенция о создании унифицированной методологии определения таможенной стоимости товаров  --- Брюссельская таможенная стоимость, которая определяется как нормальная цена товара (цена, складывающаяся между независимыми друг от друга продавцом и покупателем в условиях полной конкуренции, открытого рынка, определенная в условиях CIF (Cost, Insurance and Freight --- Стоимость, страхование и фрахт) в месте пересечения таможенной границы страны --- импортера товара).

В 1979 г. в рамках шестого Токийского раунда многосторонних торговых переговоров ГАТТ (1973--1979) было подписано Соглашение о применении ст. VII ГАТТ <<Оценка товаров для таможенных целей>>.

Заключительным этапом по унификации системы таможенной оценки стали переговоры в процессе Уругвайского раунда (1985--1994). В результате в Марракеше 15 апреля 1994 г. был подписан заключительный акт --- Соглашение по применению статьи VII ГАТТ. Данное соглашение также известно как Кодекс о таможенной стоимости ГАТТ, на который было сориентировано российское таможенное законодательство в соответствии с п. 1 ст. 12 Закона о таможенном тарифе \cites[с. 251]{mahovikova}[с. 125]{novikova}.

%Кодекс о таможенной стоимости ГАТТ закрепляет в качестве основного принципа таможенной оценки использование цены сделки (метод 1), под которой понимается цена, реально уплаченная за импортируемые товары. В цену сделки могут включаться некоторые дополнительные расходы

Таможенная стоимость товаров служит базой:
\begin{enumerate}
	\item [---] для расчета таможенной пошлины по адвалорной ставке (устанавливается в процентах от таможенной стоимости товара);
	\item [---] расчета НДС и акциза, взимаемых при ввозе товаров на таможенную территорию ТС;
	\item [---] исчисления таможенных пошлин, налогов, подлежащих уплате физическими лицами по единым ставкам;
	\item [---] определения сборов за таможенное оформление.
\end{enumerate}

Сведения о таможенной стоимости также используются для целей таможенной статистики.

Процедура определения и контроля таможенной стоимости обозначается термином <<таможенная оценка>>. Данная процедура охватывает все оценочные действия с момента первого определения таможенной стоимости товаров декларантом до принятия таможенным органом окончательного решения по таможенной стоимости, и включает заявление декларантом таможенной стоимости товаров и предоставление необходимых документов, контроль таможенной стоимости, принятие решения по таможенной стоимости, согласованного с декларантом и таможенным органом.

В соответствии с п. 3 ст. 64 ТК ТС таможенная стоимость товаров определяется декларантом либо таможенным представителем, действующим от имени и по поручению декларанта. В случаях, установленных ТК ТС --- определяется таможенным органом.

При декларировании таможенной стоимости ввозимых товаров предоставляются следующие сведения: метод определения таможенной стоимости товаров, величина таможенной стоимости товаров, обстоятельства и условия внешнеэкономической сделки, имеющие отношение к определению таможенной стоимости товаров, предоставляются подтверждающие документы.

Для заявления таможенной стоимости ввозимых в РФ товаров применяются специальные формы декларации таможенной стоимости:
\begin{enumerate}
	\item [---] ДТС-1 --- для метода 1;
	\item [---] ДТС-2 --- для методов 2--6.
\end{enumerate}

На вывозимые с территории ТС товары применяются формы деклараций ДТС-3 и ДТС-4.

Декларация таможенной стоимости --- таможенный документ установленной формы, в которой заявляются сведения о таможенной стоимости товаров по выбранному методу оценки, условиях продажи и поставки товаров, которые могут повлиять на цену сделки, а следовательно, и на таможенную стоимость. На первом (лицевом) листе приводятся общие сведения о продавце, покупателе и сделке между ними. На втором листе указывают сведения о таможенной стоимости товара и ее элементах.Дополнительные листы заполняются, если в декларируемой партии товаров больше двух наименований.

Декларация таможенной стоимости входит в состав таможенной  декларации и не действительна без нее. Все листы подписываются декларантом, каждое исправление должно быть заверено подписью декларанта.

Существуют случаи, когда заполнение декларации таможенной стоимости не требуется. К таким относятся, например, ввоз товаров физическими лицами для личных нужд, домашних, семейных или иных нужд, не связанных с предпринимательской деятельностью; если при соблюдении требований законодательства ТС не возникает обязанность по уплате таможенных пошлин, налогов, исходя из заявленной величины таможенной стоимости, и др \cite[с. 126--127]{novikova}.

Основным методом оценки ввозимых на территорию ТС товаров является метод по цене сделки с ввозимыми товарами. В соответствии ст. 19 Закона о таможенном тарифе таможенной стоимостью ввозимого на таможенную территорию РФ товара является цена сделки, фактически уплаченная или подлежащая уплате за ввозимый товар на момент пересечения им таможенной границы РФ (до порта или иного места ввоза).

Данный метод необходимо использовать при проведении оценки товаров в максимально возможной степени. Переходить к следующим методам разрешается только при установлении факта, что цена сделки отсутствует, либо не может быть определена, либо не может использоваться для определения таможенной стоимости.

Для применения метода 1 необходимо соблюдение четырех условий:
\begin{enumerate}
	\item [---] отсутствие ограничений в отношении прав покупателя на пользование и распоряжение товарами (исключением являются ограничения, которые установлены совместным решением органов ТС, ограничивают географический регион, в котором товары могут быть перепроданы, существенно не влияют на стоимость товаров);
	\item [---] продажа товаров или их цена не зависят от каких-либо условий или обязательств, влияние которых на цену товаров не может быть количественно определено;
	\item [---] никакая часть дохода или выручки от последующей продажи, распоряжения иным способом или использования товаров покупателем не причитается прямо или косвенно продавцу, кроме случаев, когда могут быть произведены дополнительные начисления;
	\item [---] покупатель и продавец не являются взаимосвязанными лицами, или являются взаимосвязанными лицами, но это не повлияло на стоимость сделки с ввозимыми товарами и доказано декларантом.
\end{enumerate}

Для признания покупателя и продавца взаимосвязанными лицами необходимо выполнение хотя бы одного из следующих условий:
\begin{enumerate}
	\item [---] являются совладельцами компании;
	\item [---] связаны торговыми отношениями;
	\item [---] совместно контролируют третье лицо либо находятся под его непосредственным контролем;
	\item [---] подконтрольны один другому;
	\item [---] их должностные лица или они сами являются родственниками;
	\item [---] должностное лицо одного участника является должностным лицом другого;
	\item [---] один участник владеет паем или акциями с правом голоса, составляющим не менее 5\% в уставном капитале другого участника.
\end{enumerate}

В соответствии со ст. 19 Закона о таможенном тарифе, ст. 5 Соглашения об определении таможенной стоимости товаров, перемещаемых через таможенную границу ТС (в ред. Протокола от 23 апреля 2012 г.), при использовании метода по цене сделки для определения таможенной стоимости в нее включаются следующие расходы, если они не были ранее в нее включены:
\begin{enumerate}
	\item [а)] расходы по доставке товара до авиапорта, порта или иного места ввоза товара на таможенную территорию РФ: стоимость транспортировки, страховая сумма, расходы на погрузку, разгрузку, перегрузку и перевалку товаров;
	\item [б)] расходы, понесенные покупателем: комиссионные и брокерские вознаграждения (кроме комиссионных по закупке товаров), стоимость упаковки и работ по упаковке, стоимость многооборотной тары, если согласно ТН ВЭД она рассматривается как единое целое с оцениваемыми товарами;
	\item [в)] соответствующая часть стоимости следующих товаров и услуг, которые прямо или косвенно были предоставлены покупателем бесплатно или по сниженной цене для использования в связи с производством и продажей на вывоз оцениваемых товаров: сырья, материалов, деталей, полуфабрикатов, и других комплектующих изделий, являющихся составной частью оцениваемых товаров; инструментов, штампов, форм и других подобных предметов, использованных при производстве оцениваемых товаров; материалов, израсходованных при производстве оцениваемых товаров (смазочных материалов, топлива и др.); инженерной проработки, опытно-конструкторской работы, дизайна, художественного оформления, эскизов и чертежей, выполненных вне территории РФ и непосредственно необходимых для производства оцениваемых товаров;
	\item [г)] лицензионные и иные платежи за использование объектов интеллектуально собственности, которые покупатель должен прямо или косвенно осуществить в качестве условия продажи оцениваемых товаров;
	\item [д)] величина части прямого или косвенного дохода продавца от любых последующих перепродаж, передачи или использования оцениваемых товаров на территории РФ.
\end{enumerate}

Приведенные расходы добавляются к цене и отражаются в специальном разделе декларации таможенной стоимости путем соответствующего дополнительного начисления. Эти поправки и доначисления должны учитываться при соблюдении следующих условий:
\begin{enumerate}
	\item [а)] данные расходы или платежи действительно имеют место и подтверждаются документально, т.е. должны выполняться на основе объективных и подлежащих количественной оценке данных. Если же такое количественное определение этих данных не представляется возможным, то таможенная стоимость не может быть определена по методу 1;
	\item [б)] эти платежи еще не включены в цену товара;
	\item [в)] эти платежи оплачиваются покупателем.
\end{enumerate}

Из цены сделки могут быть исключены суммы расходов, фактически понесенных за операции по доставке товара после ввоза на таможенную территорию РФ (от места ввоза до места доставки), при наличии документального подтверждения этих расходов \cites[с. 260--263]{mahovikova}[с. 528]{pokrovskaya}.

Таким образом, расчет таможенной стоимости по цене сделки с ввозимыми товарами производится по формуле:
\[\text{С}_\text{Т} = \text{А} + \text{Б} - \text{В}, \] где $\text{С}_\text{Т}$ --- таможенная стоимость; $\text{А}$ --- основа для расчета (цена сделки, косвенные платежи); $\text{Б}$ --- дополнительные начисления к цене сделки и подлежащие включению в таможенную стоимость; $\text{В}$ --- списываемые суммы.

Следующим по очередности применения является метод определения таможенной стоимости по цене сделки с идентичными товарами. Метод применяется в случаях, когда невозможно определить стоимость сделки с ввозимыми товарами, либо условия применения первого метода не выполняются:
\begin{enumerate}
	\item [а)] ввозимые товары не являются предметом продажи;
	\item [б)] продажа товаров связана с ограничениями, касающимися пользования или распоряжения ввозимыми товарами;
	\item [в)] продажа сопровождалась определенными условиями, вследствие чего реальная стоимость товаров не может быть определена, либо отсутствует необходимая информация для расчета соответствующих поправок к цене и осуществлению ее корректировок;
	\item [г)] продажа осуществлена между взаимозависимыми сторонами и при этом зависимость сторон оказала влияние на величину цены сделки.
\end{enumerate}

Данный метод использует в качестве базы для оценки стоимости ввозимых товаров, стоимость сделки с идентичными им товарами, таможенная стоимость которых была определена по методу 1 и принята таможенными органами.

При применении метода 2 обязательным является условие, чтобы таможенная стоимость сравниваемых товаров была определена по методу 1. Так же необходимо, чтобы товары были ввезены примерно в одно и тоже время, примерно в одинаковых количествах и на одном коммерческом уровне, что и оцениваемые товары.

Если такие характеристики различаются, то цена сделки с идентичными товарами должна быть скорректирована для компенсации этих различий.

Для применения метода 2 необходимо соблюдение следующих условий:
\begin{enumerate}
	\item [1)] товары проданы для ввоза на территорию РФ;
	\item [2)] ввезены одновременно с оцениваемыми товарами или не ранее, чем за 90 дней до ввоза оцениваемых товаров;
	\item [3)] ввезены примерно в том же количестве и (или) на тех же коммерческих условиях (если эти условия не соблюдаются, декларантом делается корректировка цены с учетом этих различий, обоснованность которой документально подтверждена таможенному органу РФ).
\end{enumerate}

Идентичный товар --- товар, одинаковый во всех отношениях с оцениваемым товаром (физические характеристики, качество и репутация на рынке, страна происхождения, производитель).

Незначительные различия во внешнем виде товаров, такие, как размер, этикетки, цвет (если не является существенным ценообразующим условием), не могут служить основанием для отказа рассмотрения товаров как идентичных.

Обязательный критерий идентичности сравниваемых товаров --- страна происхождения. Товары, произведенные в одной и той же стране различными лицами могут считаться идентичными, только если декларант и таможенные органы не обладают сведениями об идентичных товарах, произведенных тем же лицом, которое изготовило оцениваемые товары.

Если при применении данного метода обнаружится несколько сделок по идентичным товарам (с разными ценами сделки), то для определения таможенной стоимости ввозимых товаров применяется самая низкая цена сделки.

Метод определения таможенной стоимости по цене сделки с однородными товарами (метод 3) является следующим по очередности применения и используется в случаях, когда условия использования методов 1 и 2 не соблюдаются.

Данный метод, как и метод 2, в качестве базы для оценки стоимости ввозимых товаров, использует цену сделки с однородными товарами, таможенная стоимость которых была определена по методу 1 и принята таможенными органами.

Однородные товары --- товары, имеющие сходные характеристики и состоящие из схожих компонентов. При определении однородности следует учитывать такие признаки, как качество, наличие товарного знака и репутация на рынке, страна происхождения, производитель.

Условия, которые необходимо соблюдать при использования метода 3:
\begin{enumerate}
	\item [а)] товары считаются однородными при условии их производства в одной стране;
	\item [б)] если не имеется однородных товаров, произведенных тем же лицом, что и оцениваемые, для оценки используются товары, произведенные другим лицом;
	\item [в)] если проектирование, опытно-конструкторские работы, художественное оформление, дизайн, эскизы и чертежи и иные аналогичные работы выполнены в РФ, такие товары не могут считаться однородными.
\end{enumerate}

При применении метода 3 необходимо руководствоваться всеми теми же правилами, что и при применении метода 2, описанными выше \cite[с. 265--270]{mahovikova}

Следующие два метода --- метод вычитания и метод сложения применяются в случаях, когда таможенная стоимость товаров не может быть определена с помощью методов 1--3. Причем здесь очередность применения методов может быть изменена по заявлению декларанта.

Метод вычитания применяется при условии, если оцениваемые либо идентичные или однородные им товары продаются в РФ в том же состоянии, в котором они ввозятся на таможенную территорию ТС.

Базой для определения таможенной стоимости принимается цена единицы товара, по которой наибольшее совокупное количество оцениваемых, идентичных либо однородных товаров продается лицам, не являющимися взаимосвязанными с лицами, осуществляющими продажу на территории РФ, в тот же или соответствующий ему период времени, в который осуществляется ввоз оцениваемых товаров. При этом производятся вычеты следующих сумм:
\begin{enumerate}
	\item [1)] вознаграждений агенту, обычно выплачиваемых или подлежащих выплате, либо надбавок к цене, обычно производимых для получения прибыли и покрытия общих расходов в связи с продажей ввозимых товаров того же класса и вида;
	\item [2)] суммы ввозных таможенных пошлин, налогов и сборов, подлежащих уплате в соответствии с законами государства соответствующей стороны в связи с ввозом или продажей товаров;
	\item [3)] расходы, понесенные на территории РФ на транспортировку, страхование, погрузочные и разгрузочные работы.
\end{enumerate}

Для использования цены продажи оцениваемых идентичных или однородных товаров в качестве основы для определения таможенной стоимости эта продажа должна отвечать следующим условиям:
\begin{enumerate}
	\item [---] товары должны быть использованы в неизменном состоянии;
	\item [---] ввезенные товары должны продаваться одновременно с ввозом оцениваемых товаров или во время, достаточно близкое ко времени ввоза оцениваемых товаров;
	\item [---] покупателем не должны поставляться продавцу бесплатно товары или услуги, использованные для производства и продажи для ввоза на таможенну территорию импортируемых товаров.
\end{enumerate}

Если в результате обработки товары потеряли свои прежние свойства либо после обработки товар составляет весьма незначительную часть конечного продукта, метод вычитания не применяется \cites[с. 271--275]{mahovikova}[с. 130--131]{novikova}.

Метод 5 --- метод на основе сложения. Здесь рассматриваются затраты на производство ввезенных товаров и на этой основе рассчитывается их стоимость. Информацию об издержках производства можно получить только за пределами страны ввоза.

Применяя данный метод оценки в качестве базы для определения таможенной стоимости ввозимого товара использует цену товара, рассчитанную путем сложения затрат:
\begin{enumerate}
	\item [---] на изготовление или приобретение материалов, на производство, иные операции, связанные с производством оцениваемых товаров;
	\item [---] характерных для продажи на таможенную территорию ТС из страны экспортера товаров того же вида, т том числе расходов на транспортировку, погрузочные и разгрузочные работы, страхование до места пересечения таможенной границы ТС;
	\item [---] прибыли, обычно получаемой экспортером в результате поставки на таможенную территорию ТС товаров.
\end{enumerate}

Использование данного метода в соответствии с нормами международного права сопряжено с приведением ссылок на общепринятые правила ведения бухучета. Это накладывает на применение данного метода ограничения, поскольку доступ к отчетности производителя практически невозможен.

Проверка данных отчетности может быть осуществлена только в стране-изготовителе при условии, что правительство не возражает против такой проверки. Учитывая это, метод 5 должен быть объектом тщательной проверки.

























