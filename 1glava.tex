\section{Таможенная стоимость, методы ее определения.}

Вопрос определения и контроля таможенной стоимости ввозимых товаров в настоящее время является одним из самых актуальных, так как таможенная стоимость служит основой для начисления таможенных платежей.

Правильное определение таможенной стоимости является особо важным для таможенных органов, что связано с тем, что таможенные платежи составляют существенную часть федерального бюджета. Для предприятий вопрос определения таможенной стоимости играет ключевую роль, так как эффективность внешнеэкономической деятельности во многом зависит от уплачиваемых таможенных платежей \cite[с. 249--250]{mahovikova}.

Таможенная стоимость товаров, ввозимых на территорию таможенного союза (далее ТС), определяется в соответствии с международным договором государств --- членов ТС, регулирующим вопросы определения таможенной стоимости товаров, перемещаемых через границу. В РФ определяется в соответствии Соглашением между Правительством РФ, Правительством Республики Беларусь и Правительством Республики Казахстан от 25.01.2008 <<Об определении таможенной стоимости товаров, перемещаемых через таможенную границу Таможенного союза>> (ред. от 23.04.1012). Таможенная стоимость товаров, вывозимых с таможенной территории ТС, определяется в соответствии с законодательством государства --- члена ТС, таможенному органу которого производится декларирование товаров \cite[с. 486]{tolkushkin}.

Таможенная стоимость товара --- стоимость сделки с товаром, т.е. цена, фактически уплаченная или подлежащая уплате за эти товары при их продаже для вывоза на единую  таможенную территорию ТС. Таможенная стоимость --- это особая разновидность стоимости товара, в определении которой помимо продавца и покупателя участвует третья сторона --- таможенный орган.

Основу современной процедуры таможенной оценки товаров, принятой в большинстве стран мира, составляют ст. VII ГАТТ (General Agreement on Tariffs and Trade --- Генеральное соглашение по тарифам и торговле) и Соглашение по применению статьи VII ГАТТ. Наиболее точное определение таможенной стоимости дается в Соглашении по применению ст. VII ГАТТ \cite[с. 124]{novikova}.
\begin{quote}
	Таможенная стоимость товара --- цена, фактически уплаченная или подлежащая уплате за товары при продаже с целью экспорта в страну импорта, скорректированная с учетом установленных дополнительных начислений к этой цене.
\end{quote}

Необходимость унификации процессов определения таможенной стоимости была связана с тем, что длительное время в разных странах применяемые методы оценки таможенной стоимости значительно отличались. Не зная конечную цену продаваемого товара невозможно было определить экономическую эффективность сделки.

В 1950 г. в Брюсселе была заключена Конвенция о создании унифицированной методологии определения таможенной стоимости товаров  --- Брюссельская таможенная стоимость, которая определяется как нормальная цена товара (цена, складывающаяся между независимыми друг от друга продавцом и покупателем в условиях полной конкуренции, открытого рынка, определенная в условиях CIF (Cost, Insurance and Freight --- Стоимость, страхование и фрахт) в месте пересечения таможенной границы страны --- импортера товара).

В 1979 г. в рамках шестого Токийского раунда многосторонних торговых переговоров ГАТТ (1973--1979) было подписано Соглашение о применении ст. VII ГАТТ <<Оценка товаров для таможенных целей>>.

Заключительным этапом по унификации системы таможенной оценки стали переговоры в процессе Уругвайского раунда (1985--1994). В результате в Марракеше 15 апреля 1994 г. был подписан заключительный акт --- Соглашение по применению статьи VII ГАТТ. Данное соглашение также известно как Кодекс о таможенной стоимости ГАТТ, на который было сориентировано российское таможенное законодательство в соответствии с п. 1 ст. 12 Закона о таможенном тарифе \cites[с. 251]{mahovikova}[с. 125]{novikova}.

%Кодекс о таможенной стоимости ГАТТ закрепляет в качестве основного принципа таможенной оценки использование цены сделки (метод 1), под которой понимается цена, реально уплаченная за импортируемые товары. В цену сделки могут включаться некоторые дополнительные расходы

Таможенная стоимость товаров служит базой:
\begin{enumerate}
	\item [---] для расчета таможенной пошлины по адвалорной ставке (устанавливается в процентах от таможенной стоимости товара);
	\item [---] расчета НДС и акциза, взимаемых при ввозе товаров на таможенную территорию ТС;
	\item [---] исчисления таможенных пошлин, налогов, подлежащих уплате физическими лицами по единым ставкам;
	\item [---] определения сборов за таможенное оформление.
\end{enumerate}

Сведения о таможенной стоимости также используются для целей таможенной статистики.

Процедура определения и контроля таможенной стоимости обозначается термином <<таможенная оценка>>. Данная процедура охватывает все оценочные действия с момента первого определения таможенной стоимости товаров декларантом до принятия таможенным органом окончательного решения по таможенной стоимости, и включает заявление декларантом таможенной стоимости товаров и предоставление необходимых документов, контроль таможенной стоимости, принятие решения по таможенной стоимости, согласованного с декларантом и таможенным органом.

В соответствии с п. 3 ст. 64 ТК ТС таможенная стоимость товаров определяется декларантом либо таможенным представителем, действующим от имени и по поручению декларанта. В случаях, установленных ТК ТС --- определяется таможенным органом.

При декларировании таможенной стоимости ввозимых товаров предоставляются следующие сведения: метод определения таможенной стоимости товаров, величина таможенной стоимости товаров, обстоятельства и условия внешнеэкономической сделки, имеющие отношение к определению таможенной стоимости товаров, предоставляются подтверждающие документы.

Для заявления таможенной стоимости ввозимых в РФ товаров применяются специальные формы декларации таможенной стоимости:
\begin{enumerate}
	\item [---] ДТС-1 --- для метода 1;
	\item [---] ДТС-2 --- для методов 2--6.
\end{enumerate}

На вывозимые с территории ТС товары применяются формы деклараций ДТС-3 и ДТС-4.

Декларация таможенной стоимости --- таможенный документ установленной формы, в которой заявляются сведения о таможенной стоимости товаров по выбранному методу оценки, условиях продажи и поставки товаров, которые могут повлиять на цену сделки, а следовательно, и на таможенную стоимость. На первом (лицевом) листе приводятся общие сведения о продавце, покупателе и сделке между ними. На втором листе указывают сведения о таможенной стоимости товара и ее элементах.Дополнительные листы заполняются, если в декларируемой партии товаров больше двух наименований.

Декларация таможенной стоимости входит в состав таможенной  декларации и не действительна без нее. Все листы подписываются декларантом, каждое исправление должно быть заверено подписью декларанта.

Существуют случаи, когда заполнение декларации таможенной стоимости не требуется. К таки








