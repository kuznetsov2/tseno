\section{Таможенная стоимость, методы ее определения.}

Вопрос определения и контроля таможенной стоимости ввозимых товаров в настоящее время является одним из самых актуальных, так как таможенная стоимость служит основой для начисления таможенных платежей.

Правильное определение таможенной стоимости является особо важным для таможенных органов, что связано с тем, что таможенные платежи составляют существенную часть федерального бюджета. Для предприятий вопрос определения таможенной стоимости играет ключевую роль, так как эффективность внешнеэкономической деятельности во многом зависит от уплачиваемых таможенных платежей \cite[с. 249--250]{mahovikova}.

Таможенная стоимость товаров, ввозимых на территорию таможенного союза (далее ТС), определяется в соответствии с международным договором государств --- членов ТС, регулирующим вопросы определения таможенной стоимости товаров, перемещаемых через границу. Таможенная стоимость товаров, вывозимых с таможенной территории ТС, определяется в соответствии с законодательством государства --- члена ТС, таможенному органу которого производится декларирование товаров \cite[с. 486]{tolkushkin}.

Таможенная стоимость товара --- стоимость сделки с товаром, т.е. цена, фактически уплаченная или подлежащая уплате за эти товары при их продаже для вывоза на единую  таможенную территорию ТС. Таможенная стоимость --- это особая разновидность стоимости товара, в определении которой помимо продавца и покупателя участвует третья сторона --- таможенный орган.

Основу современной процедуры таможенной оценки товаров, принятой в большинстве стран мира, составляют ст. VII ГАТТ и Соглашение по применению статьи VII ГАТТ. Наиболее точное определение таможенной стоимости дается в Соглашении по применению ст. VII ГАТТ \cite[с. 124]{novikova}.
\begin{quote}
	Таможенная стоимость товара --- цена, фактически уплаченная или подлежащая уплате за товары при продаже с целью экспорта в страну импорта, скорректированная с учетом установленных дополнительных начислений к этой цене.
\end{quote}



























