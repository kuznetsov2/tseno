\section{Таможенная стоимость, методы ее определения.}

Вопрос определения и контроля таможенной стоимости ввозимых товаров в настоящее время является одним из самых актуальных, так как таможенная стоимость служит основой для начисления таможенных платежей.

Правильное определение таможенной стоимости является особо важным для таможенных органов, что связано с тем, что таможенные платежи составляют существенную часть федерального бюджета. Для предприятий вопрос определения таможенной стоимости играет ключевую роль, так как эффективность внешнеэкономической деятельности во многом зависит от уплачиваемых таможенных платежей \cite[с. 249--250]{mahovikova}.

Таможенная стоимость товаров, ввозимых на территорию таможенного союза (далее ТС), определяется в соответствии с международным договором государств --- членов ТС, регулирующим вопросы определения таможенной стоимости товаров, перемещаемых через границу. Таможенная стоимость товаров, вывозимых с таможенной территории ТС, определяется в соответствии с законодательством государства --- члена ТС, таможенному органу которого производится декларирование товаров \cite[с. 486]{tolkushkin}.





























